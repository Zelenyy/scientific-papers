\section{Анализ структуры грозового облака на основе наблюдения TGE}\label{sec:thunderstorm/llletge} 
В отличии от TGF представляющие собой очень короткие (~10 мкс) вспышки гамма- и рентгеновского излучения из грозового облака, TGE представляет собой длительное повышение гамма-фона. По результатам наблюдений на горе Арагац академиком Чилингаряном~\cite{PhysRevD.98.082001} была предложено разделить TGE на две условные группы: High Energy Particle TGE (HEP TGE) имеющие длительность порядка нескольких минут и значительную высокоэнергичную (энергия гамма-квантов более 3 МэВ) компоненту и Long-Lasting Low Energy TGE (LLLL TGE) имеющие длительность более двух часов и при этом содержащие в основном низкоэнергитичные (менее 3 МэВ) гамма-кванты. При этом можно отметить следующие особенности: 
\begin{itemize}
    \item HEP TGE  возникает при глубоких  провалах  приповерхностного поля, но не каждый провал сопровождается HEP TGE,
    \item Возможно существование LLLE TGE без  HEP TGE,
    \item Возможно cуществование LLLE TGE при полях ниже поля убегания электронов,
    \item HEP TGE может возникать несколько раз: оно может быть разрушено из-за вспышки молнии и потом восстановится.
\end{itemize}
В контексте изучения возникновения молнии HEP TGE и LLLE TGE интересны, так как они могут помочь описать структуру грозового облака.  Так первым представление об грозовом облаке был вертикальный диполь, однако реальная его структура гораздо сложнее, в данной работе мы рассмотрим генерацию TGE c точки зрения трипольной модели(хотя по всей видимости облака могут иметь и большее число слоев). При представлении облака как вертикального триполя считается что в центре облака собирается отрицательный заряд, в верхней части облака и на земле индуцируется положительный заряд, и наконец в нижней части облака формируется небольшая область с положительным зарядом. Тогда можно сформулировать следующие гипотезы:
\begin{itemize}
    \item Затравочными частицами служат электроны от космических лучей;
    \item LLLE TGE -возникает за счет поля между основным отрицательным зарядом и зарядом индуцированным в земле;
    \item По мере развития нижнего положительного заряда (НПЗ)  возрастает поле и поток частиц в LLLE TGE переходит в HEP TGE;
    \item Регистрация HEP TGE происходит при прохождении НПЗ над детектором
\end{itemize}
Для проверки этих гипотез в работе были проанализированны экспериментальные наблюдения которые сравнивались с моделированием с помошью CORSIKA и GEANT4. Далее приводятся результаты проведенного для работы~\cite{PhysRevD.98.082001}  автором диссертации GEANT4 моделирования. В частности автором были проведенным моделирования превышения гамма-квантов в зависимости от электрического поля, а также получены их угловые и радиальные распределения. Симуляция проводилась при следующих параметрах: первичными частиц являются гамма-кванты в диапазоне энергий от 1 до 100 МэВ, имеющие степенной спектр вида $E^{-1.42}$, начальная точка расположена в тысяче метрах над станцией(которая находится на высоте 3200 метров над уровнем моря), моделирование проводилось в подкритических полях (значение критического поля для данных высот порядка 1.7-1.8 кВ/см
), также моделирование проводилось с использование двух физических листов G4EmPhysStandard и G4EmPhysStandar\_opt4 (использовался GEANT4 версии 4.10.4). График демонстрирует рост потока гамму, даже в подкритических полях и взрывной рост при приближении к порогу. Левый график демонстрирует радиальное расхождение гамма-квантов от RREA вследствии комптон-эффекта, как мы видим маловероятно наблюдать гамма-кванты на растоянии большем километра от центра лавины, кроме того для гамма-квантов рассеявшихся на растояние более километра, максимум по углу смещен в горизонтальный сторону, что затрудняет их регистрацию. Результаты моделирования автора согласуются с другим моделированием проведенным в программе CORSIKA, и на основе анализа приведенного в работе~\cite{PhysRevD.98.082001} можно утверждать что сформулированные гипотезы выполняются и можно сделать вывод что наблюдаемые данные по TGE могут быть объяснены трипольной структурой облака поля и наблюдение за TGE можно использовать для наблюдения эволюции нижнего заряженного слоя, его образования, перемещения и разрушения.