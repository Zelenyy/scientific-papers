\chapter*{Список сокращений и условных обозначений} % Заголовок
\addcontentsline{toc}{chapter}{Список сокращений и условных обозначений}  % Добавляем его в оглавление
\noindent
%\begin{longtabu} to \dimexpr \textwidth-5\tabcolsep {r X}
\begin{longtabu} to \textwidth {r X}

%General
\textbf{GEANT4} & солнечные космические лучи\\

% Satellite
\textbf{СКЛ} & солнечные космические лучи\\
\textbf{ГКЛ} & галактические космические лучи\\
\textbf{GLE} & ground level events, события наземных возрастаний\\
\textbf{КВМ} & корональный выброс массы\\
\textbf{КА} & космический аппарат\\
% Thunderstorm
\textbf{TGE} & \\
\textbf{TGF} & \\
\textbf{ПУЭ} & пробой на убегающих электронах\\

\end{longtabu}
\addtocounter{table}{-1}% Нужно откатить на единицу счетчик номеров таблиц, так как предыдующая таблица сделана для удобства представления информации по ГОСТ
