\chapter{Монте-Карло моделирование с помощью транспортного кода GEANT4}\label{ch:theory}

\section{Монте-Карло моделирование с GEANT4
}\label{sec:theory/geant4}

\section{Взаимодействия фотонов и заряженных частиц с веществом
}\label{sec:theory/propagation}

Кратко опишем какие физические процессы происходят с гамма-квантами, протонами, электронами и позитронами происходят в интересующем нас диапазоне энергий до 150 МэВ.

\subsection{Взаимодействие электромагнитного излучения с веществом}

~\cite{bespalov2008, kolchuzkin1978, nemec1975}

\noindent Маркированный список:
\begin{itemize}
    \item Когерентное рассеяние
    \item Комптоновское рассеяние
    \item Фотоэффект
    \item Эффект образования  электрон-позитронных пар
    \item Фотоядерные реакции
\end{itemize}

\subsection{Упругие столкновения заряженных частиц в с веществом}

\subsection{Неупругие столкновения заряженных частиц с веществом}

\subsection{Излучение заряженных частиц движущихся в веществе}


  

\section{Электромагнитные модели в GEANT4 }\label{sec:theory/models}

EmStandartOptx, PENELOPA, Livermore

\subsection{PENELOPA}
Для расчета сечений используется и экспериментальные данные, и аналитические формулы.Описывает процессы для фотонов, электронов и позитронов: рэлеевское рассеяние, комптон-эффект, тормозное излучение, фотоэффект, рождение электрон-позитронных пар гаммой, ионизация (в разработке), аннигиляция позитрона (в разработке)
Рекомендованные энергетические пределы: 100 эВ – 1 ГэВ
Рассматриваются тормозное излучение и ионизация для позитрона, а также его аннигиляция

\subsection{Livermore}
Сечение рассчитывается на основе экспериментальных данных
Описывает процессы для фотонов, электронов, ионов и адронов: рэлеевское рассеяние, комптон-эффект, тормозное излучение, фотоэффект, рождение электрон-позитронных пар гаммой, ионизация, флюоресценция, электронная эмиссия Ожэ возбужденных атомов
Рекомендованные энергетические пределы: 250 эВ – 100 ГэВ

- Рэлеевское рассеяние ~ 10 эВ

- Комптон-эффект ~ 250 эВ

- Тормозное излучение ~ 10 эВ

- Ионизация ~ 100 эВ

- Фотоэффект ~ 10 эВ

- Рождение электрон-позитронных пар гаммой ~ 1 МэВ

Не учитываются позитроны (кроме рождения электрон-позитронных пар)

\subsection{}
