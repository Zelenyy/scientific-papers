\chapter{Монте-Карло моделирование с помощью транспортного кода GEANT4}\label{ch:theory}

\section{Монте-Карло моделирование с GEANT4
}\label{sec:theory/geant4}

\section{Взаимодействия фотонов и заряженных частиц с веществом
}\label{sec:theory/propagation}

Кратко опишем какие физические процессы происходят с гамма-квантами, протонами, электронами и позитронами происходят в интересующем нас диапазоне энергий до 150 МэВ. Взаимодействие гамма-квантов описывается пятью процессами: когерентное рассеяние, комптоновское рассеяние, фотоэффект, рождение электрон-позитронных пар, фотоядерные реакции. Каждый из этих процессов по своему зависит от состава среды в которой распространяются гамма-кванты, кроме того учет вклада того или иного процесса зависит от рассматриваемого диапазона энергий. Так когерентное рассеяние важно в оптическом и ультрафиолетовом диапазоне, и не дает значительного вклад в рентгеновском, а в гамма-диапазоне им можно полностью пренебречь. Следующие три процессы являются основными для нас, поскольку именно они определяют поглощение и рассеяние гамма-квантов. Фотоядерные реакции имеют меньшие сечения, а также обычно имеют высокий порог реакции, поэтому изучение их вклада актуально для больших потоков гамма-квантов с энергиями большим 8 МэВ, а также в том случае когда нам интересно возникновение нейтронной компоненты.
При рассмотрении процессов связанных с движение протонов и электронов удобно разделить эти процессы на упругие и неупругие. К неупругим процессам относятся столкновения с атомными электронами, приводящие к возбуждению и ионизации атомов и молекул, рассеяние с излучение тормозных фотонов и ядерные взаимодействия. 
Для протонов и электронов большую роль имеют процессы связанные с потерями энергии на ионизацию вещества и радиационные потери.  


%\subsection{Взаимодействие электромагнитного излучения с веществом}

~\cite{bespalov2008, kolchuzkin1978, nemec1975}

%\noindent Маркированный список:
%\begin{itemize}
%    \item Когерентное рассеяние
%    \item Комптоновское рассеяние
%    \item Фотоэффект
%    \item Эффект образования  электрон-позитронных пар
%    \item Фотоядерные реакции
%\end{itemize}

%\subsection{Упругие столкновения заряженных частиц в с веществом}

%\subsection{Неупругие столкновения заряженных частиц с веществом}

%\subsection{Излучение заряженных частиц движущихся в веществе}


  

\section{Электромагнитные модели в GEANT4 }\label{sec:theory/models}


Для моделирования электромагнитных процессов в GEANT4 используются специальный конструкторы, которые надо добавить в физический лист моделирования. Эти конструкторы как правило отличаются моделями используемыми в конкретных диапазонах энергий, а так же настройками связанными с точностью численного моделирования. В таблице приведены физические модели используемые в различных моделях для инересующих нас частиц.
%
%\newpage
%
%\begingroup
%\centering
%\small
%\begin{longtable}[c]{|l|c|c|c|c|}
%    \caption{Наименование таблицы средней длины}\label{tab:test5}% label всегда желательно идти после caption
%    \\[-0.45\onelineskip]
%    \hline
%    Конструктор & Когеретное рассеяние & Фотоэффект & Комптон-эффект & Рождение пар\\ \hline
%    \endfirsthead%
%    \caption*{\tabcapalign Продолжение таблицы~\thetable}\\[-0.45\onelineskip]
%    \hline
%    Конструктор & Когеретное рассеяние & Фотоэффект & Комптон-эффект & Рождение пар\\ \hline
%    \endhead
%    \hline
%    \endfoot
%    \hline
%    \endlastfoot
%    \multicolumn{4}{|l|}{\&INP}        \\ \hline
%    \hline 
%     EM Opt0 &  &  &  & \\
%     \hline 
%     EM Opt1 & - &  &  & \\
%     \hline 
%     EM Opt2 & - &  &  & \\
%     \hline 
%     EM Opt3 & d &  &  & \\
%     \hline 
%     EM Opt4 & d &  &  & \\
%     \hline 
%     EM Liv &  d&  &  & \\
%     \hline 
%     EM Pen & PenRayleigh 0-100 GeV &  &  & \\
%     \hline 
%     EM GS & d &  &  & \\
%     \hline 
%     EM LE & d &  &  & \\
%     \hline 
%     EM WVI & d &  &  & \\
%     \hline 
%     EM SS & d &  &  & \\
%     \hline 
%     EM DNA & d &  &  & \\
%\end{longtable}
%\normalsize% возвращаем шрифт к нормальному
%\endgroup
%\newpage

%EmStandartOptx, PENELOPA, Livermore
%
%\subsection{PENELOPA}
%Для расчета сечений используется и экспериментальные данные, и аналитические формулы.Описывает процессы для фотонов, электронов и позитронов: рэлеевское рассеяние, комптон-эффект, тормозное излучение, фотоэффект, рождение электрон-позитронных пар гаммой, ионизация (в разработке), аннигиляция позитрона (в разработке)
%Рекомендованные энергетические пределы: 100 эВ – 1 ГэВ
%Рассматриваются тормозное излучение и ионизация для позитрона, а также его аннигиляция
%
%\subsection{Livermore}
%Сечение рассчитывается на основе экспериментальных данных
%Описывает процессы для фотонов, электронов, ионов и адронов: рэлеевское рассеяние, комптон-эффект, тормозное излучение, фотоэффект, рождение электрон-позитронных пар гаммой, ионизация, флюоресценция, электронная эмиссия Ожэ возбужденных атомов
%Рекомендованные энергетические пределы: 250 эВ – 100 ГэВ
%
%- Рэлеевское рассеяние ~ 10 эВ
%
%- Комптон-эффект ~ 250 эВ
%
%- Тормозное излучение ~ 10 эВ
%
%- Ионизация ~ 100 эВ
%
%- Фотоэффект ~ 10 эВ
%
%- Рождение электрон-позитронных пар гаммой ~ 1 МэВ
%
%Не учитываются позитроны (кроме рождения электрон-позитронных пар)
%
%\subsection{}
