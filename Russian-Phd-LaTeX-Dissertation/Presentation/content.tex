%\section{Списки}
\begin{frame}[plain, noframenumbering]
    \begin{center}
        \Huge
         Проектирование детектора солнечных космических лучей
    \end{center}
\end{frame}
%
%\subsection{Нумерованные}
%
\begin{frame}
    \frametitle{Проектирование детектора солнечных космических лучей}
                \begin{columns}
    	\begin{column}{0.6\textwidth}
    		\textbf{Цели:}
    		\begin{itemize}
    			\item Исследование солнечных космических лучей и
    			солнечных вспышек;
    			\item Обеспечение радиационной безопасности для космонавтов
    			и электроники;
    		\end{itemize}
    		\textbf{Задачи:}
    		\begin{itemize}
    			\item Разработка и оптимизация схемы детектора;
    			\item Методики анализа в разных режимах работы;
    		\end{itemize}
    	\end{column}
    	\begin{column}{0.4\textwidth}
    		\textbf{Требования:}
    	\begin{itemize}
    			\item Протоны от 10 МэВ до 100 МэВ;
    			\item Электроны от 1 МэВ до 10 МэВ;
    			\item Загрузки от $10^6$ Гц;
    			\item Канал связи 1-10 МБ/сутки;
    			\item Размеры  8 см х 6 см х 6 см;
    			\item Масса до 700 грамм.
    	\end{itemize}
    	\end{column}
    \end{columns} 
\end{frame}

\begin{frame}
\frametitle{Проектирование детектора солнечных космических лучей}
\begin{columns}
	\begin{column}{0.4\textwidth}
		\textbf{Сцинтилляционный cегментированный детектор c SiPM:}
	\end{column}
	\begin{column}{0.6\textwidth}
		\begin{itemize}
			\item Определение энергии и типа частиц по распределению ионизационных потерь по глубине детектора.
			\item Одночастичный (счетный) режим работы для загрузок менее $10^4 – 10^5$ Гц
			\item Интегральный режим (при загрузках $>10^6$ Гц и узких каналах связи)
		\end{itemize}
%		\includegraphics[width=1\textwidth]{}
	\end{column}
\end{columns} 
\end{frame}



\begin{frame}[plain, noframenumbering]
\begin{center}
	\Huge
	Моделирования сканера транспортных контейнеров
\end{center}
\end{frame}

\begin{frame}
\frametitle{Моделирования сканера транспортных контейнеров}
\begin{columns}
	\begin{column}{0.7\textwidth}
		\begin{block}{Поглощение гамма-квантов}
		$$
		T(E_0, t, Z) = \frac{\int \limits_0^{E_0} S(E_0, E) \exp(-\mu(E,Z)\times t)~dE)}{\int \limits_0^{E_0} S(E_0, E)~dE}
		$$
	\end{block}
		%            \hline
%		\includegraphics[width=1\textwidth]{}
    \begin{block}{Метод дуальных энергий}
    	$$
    	F(z) = \frac{|t(E^{(1)}_0,z) - t(E^{(2)}_0,z)|}{t(E^{(1)}_0,z)} \to min
    	$$
%	\begin{itemize}
%		\item We can't defined Z if optical thickness $t$ is unknown.
%		\item But we can use two electron beams with different energy which give gamma rays with up-limit energy $E^{(1)}_0$ and $E^{(2)}_0$.
%		\item Then we can get Z as a result of minimizing this function:
%		\item This technique allows to determine scan object as a one from four possible groups: $Z_{eff} \sim 5$, $Z_{eff} \sim 13$, $Z_{eff} \sim 26$, $Z_{eff} \sim 82$.
%	\end{itemize}
	
\end{block}
	\end{column}
	\vline~
	\begin{column}{0.4\textwidth} 
		$T$ -  transmittance\\
		$S(E_0, E)$ - response function\\
		$\mu(E,Z)$ - attenuation\\
		$t$ -  optical thickness\\
		$E_0$ -  up-limit energy of bremsstrahlung\\
		$E$ - energy of gamma ray\\
		$Z$ - charge of nuclei 
		%            \hline
	\end{column}
\end{columns}  
\end{frame}

\begin{frame}
\frametitle{Моделирования сканера транспортных контейнеров}
\begin{block}{Недостатки метода дуальных энергий}%<1>
	\begin{itemize}
		\item Усложнение конструкции облучателя.
		\item Низкая эффективность если материалы состоят из атомов с сильно различающимися зарядовыми числами.
	\end{itemize}
\end{block}
\begin{block}{Наши предложения}%<2>
	Использовать монохроматический электронный пучок, но при этом измерять энергетическое распределение гамма-квантов.
\end{block}
\begin{block}{Задачи}%<2>
	Моделирование энергетического спектра гамма-квантов при облучении контейнера.
\end{block}
\end{frame}

%\note{
%    Этот текст будет виден только если его отображение включено в файле \textbf{Presentation/setup}.
%    Для раздельного вывода презентации и заметок на разные экраны (как в impress или powerpoint) можно использовать программу \textit{pdf-presenter-console}.
%}
%
%\subsection{Не нумерованные}
%
%
%\begin{frame}
%    \frametitle{Перечисления}
%    \begin{itemize}
%        \item Проблема 1
%        \item Проблема 2
%        \item Проблема 3
%    \end{itemize}
%\end{frame}
%\note[itemize]{
%    \item Тезис 1
%    \item Тезис 2
%    \item Тезис 3
%}


