\documentclass[a4paper]{panl}
\usepackage{cite}
\usepackage{wrapfig}
\usepackage{graphicx}
\usepackage{amssymb}
\usepackage{amsfonts}
\usepackage{amsmath}
\usepackage{longtable}
\usepackage{rotating}
\usepackage{lscape}
\usepackage{epsfig}
\usepackage{multirow}
%\originalTeX
\russianTeX

\graphicspath{{figures/}}
\begin{document}
% Journal sections (see http://pkp.jinr.ru/index.php/PEPAN_LETTERS/about/editorialPolicies#focusAndScope)
%\issuearea{Physics of Elementary Particles and Atomic Nuclei. Theory}
% or in Russian
\issuearea{ФИЗИКА ЭЛЕМЕНТАРНЫХ ЧАСТИЦ И АТОМНОГО ЯДРА. ТЕОРИЯ}
\title{Анализ химического состава при сканировании транспортных контейнеров гамма-излучением \\ Chemical composition analysis for X-ray transport container scans.}
\maketitle
\authors{А. В. Зелёная$^{~a,}$\footnote{E-mail: zelyenaya.av@phystech.edu}, М. Е. Зелёный$^{~a,b,}$\footnote{E-mail: mihail.zelenyy@phystech.edu}, А. А. Туринге$^{~a}$,  В. Г. Недорезов$^{~a,}$\footnote{E-mail: vladimir@cpc.inr.ac.ru}}
\from{$^{a}$\,Институт ядерных исследований РАН}
\vspace{-3mm}
\from{$^{b}$\,Московский физико-технический институт}

\begin{abstract}
% Russian translation of the abstract
Для обеспечения национальной безопасности важен контроль перемещения опасных или стратегически важных грузов, таких как взрывчатые вещества, радиоактивные материалы, редкие и драгоценные металлы. Проводить такой контроль можно, сканируя содержимое транспортных контейнеров гамма-излучением.
 
В данной работе рассмотрена существующая методика дуальных энергий и предложен альтернативный способ, основанный на измерении энергетического распределения гамма-квантов. Для оценки было проведено моделирование с помощью транспортного кода GEANT4.  Также выполнен эксперимент по измерению энергетического разрешения детектора на основе сцинтиллирующего кристалла BGO и кремневого фотоумножителя.

\vspace{0.2cm}
It is important for national security to control the movement of dangerous or strategically cargo such as explosives, radioactive materials, rare and precious metals. This control can be provided by scanning transport containers by gamma rays.

In this report the existing technique for scanning (dual energy method) is considered and the alternative method based on measuring the energy distribution of gamma rays is proposed. For estimation perspectives of the proposing method, the  corresponding simulation was conducted by using the GEANT4 toolkit. The example of the algorithm of reconstruction the chemical composition of the scanned object is also considered. In addition the experiment for estimation energy resolution of the detector based on a scintillation crystal BGO and SiPM was carried out.\\
\end{abstract}
\vspace*{6pt}

\noindent
PACS: 02.70.$-$c; 23.20.Nx; 32.90.$+$a

\newpage
\label{sec:intro}
\section*{Введение}
Использование высокоэнергетического гамма-излучения в прикладной томографии на данный момент широко распространено.  В этой статье мы рассмотрим возможности улучшения существующего метода как путем модернизации оборудования, так и путем разработки математических алгоритмов, которые более полно используют информацию, содержащуюся в измеренных значениях. В начале мы рассмотрим существующий метод дуальных энергий и определим возможные направления для создания более точной методики. Далее в работе приведено описание нескольких симуляций и численных экспериментов с обсуждением результатов. Также в работе приведены результаты измерений характеристик сцинтилляционного кристалла BGO, который может послужить основой для модернизированного оборудования. 
\section*{Метод дуальных энергий}
\begin{figure}[t]
    \begin{center}
        \begin{minipage}[h]{0.49\linewidth}
            \center{ \includegraphics[width=60mm]{Attenuation.pdf}  \\ а)}
        \end{minipage}
        \hfill
        \begin{minipage}[h]{0.49\linewidth}
            \center{ \includegraphics[width=60mm]{Bremsstrahlung.pdf}  \\ б)}
        \end{minipage}
        \vspace{-3mm}
        \caption{а) Массовый коэффициент ослабления для различных материалов. б) Спектр тормозного излучения от электрона с энергией 10 МэВ.}
    \end{center}
    \labelf{pic:att}
    \vspace{-5mm}
\end{figure}

Рассмотрим, как уменьшается поток гамма-лучей. Коэффициент пропускания описывается следующим уравнением:
\begin{equation}
\label{eq:trans}
T(E_0, t, Z) = \frac{\int \limits_0^{E_0} S(E_0, E) \exp(-\mu(E,Z)\times t)~dE)}{\int \limits_0^{E_0} S(E_0, E)~dE},
\end{equation}
где $T$ --- прозрачность материала для гамма-излучения, $S(E_0, E)$ --- функция отклика детектора, $\mu(E,Z)$ --- массовый коэффициент ослабления, $t$ ---  оптическая толщина материала, $E_0$ --- предельная энергия тормозного излучения, $E$ --- энергия гамма-излучения, $Z$ --- заряд ядра исследуемого материала.

Предположим, что в качестве источника гамма-лучей используется тормозное излучение со спектром как на рисунке~\ref{pic:att}б и максимальной энергией $E_0$, зависящей от энергии электронного пучка. Коэффициент прозрачности  $T(E_0, t, Z)$ также зависит от среднего массового коэффициента ослабления материала. Рисунок~\ref{pic:att}а показывает зависимость коэффициента ослабления от энергии для различных материалов. Мы можем выделить три области: начальную, в которой доминирует фотоэлектрический эффект и могут быть разделены только материалы с большим зарядом ядра; среднюю, в которой доминирует комптоновское рассеяние и материалы не различимы,и наконец область, где основное влияние оказывает процесс рождения электрон-позитронных пар и материалы достаточно хорошо различимы~\cite{heitler1984quantum, ALLISON2016186, spirin}. Последняя область может быть использована для метода дуальных энергий~\cite{spirin}.

Уравнение~\ref{eq:trans} не позволяет определить материал, если неизвестна оптическая толщина материала. Для решения этой проблемы в методе дуальной энергии предлагается использовать два электронных пучка с различной энергией. Используя прозрачность для двух предельных энергий гамма-лучей $E^{(1)}_0$ и $E^{(2)}_0$, а затем минимизируя функционал
\begin{equation}
F(z) = \frac{|t(E^{(1)}_0,z) - t(E^{(2)}_0,z)|}{t(E^{(1)}_0,z)} \to min,
\end{equation}
становиться возможным исключить неизвестную нам оптическую толщину и вычислить эффективное зарядовое число для исследуемого материала. Данный метод позволяет отнести сканируемый материал к одной из четырёх групп, разделённых по эффективному зарядовому числу: $Z_{eff} \sim 5$, $Z_{eff} \sim 13$, $Z_{eff} \sim 26$, $Z_{eff} \sim 82$.\\
Однако, метод дуальной энергии имеет некоторые недостатки, среди которых мы выделим два:
    \begin{itemize}
        \item Необходимость в двух пучках различной энергии ведет к усложнению конструкции сканера.
        \item Данный метод будет иметь малую эффективность в случае материала, состоящего из сильно различающихся по заряду элементов.
    \end{itemize}
Поэтому мы предлагаем альтернативный подход:
    \begin{itemize}
        \item Использовать только один электронный пучок с энергией 10~МэВ.
        \item Измерять не только пространственное, но и энергетическое распределение гамма-квантов.
    \end{itemize}

\section*{Моделирование}
\begin{figure}[t]
    \begin{center}
        \includegraphics[width=120mm]{yed_schema_1.pdf}
        \vspace{-3mm}
        \caption{Схема симуляции}
    \end{center}
    \labelf{pic:schema1}
    \vspace{-5mm}
\end{figure}
Для оценки мы провели несколько GEANT4~\cite{ALLISON2016186} симуляций, используя схему (рис.~\ref{pic:schema1}): электронный пучок с энергией 10~МэВ сталкивается с вольфрамовым конвертором, создавая тормозное излучение, которое облучает стальной двухметровый контейнер, внутри которого находится сканируемый объект и регистрируется детектором. Расстояние между вольфрамовым конвертором и контейнером составляет два метра, между контейнером и детектором --- 10~см.

Приведем несколько примеров проведенного моделирования:
   \begin{itemize}
        \item На рисунке~\ref{pic:sword}а показан пример опасного стального объекта неоднородной толщины, сравнимой с толщиной стенок контейнера.
        \item Рисунки~\ref{pic:sword}б и~\ref{pic:hex}а показывают результат моделирования уранового кубика с ребром 6~сантиметров (вес около 4~кг), помещенного в свинцовую сферу толщиной 1~см. Как показало моделирование, такой куб можно обнаружить с толщиной оболочки до 5~сантиметров.
        \item Рисунок~\ref{pic:hex}б демонстрирует разницу между двумя органическими материалами: безопасным --- целлюлозой и опасным --- гексогеном (RDX). Разница значительна, это означает, что можно разработать алгоритмы поиска органических взрывчатых веществ.
        \item На рисунке~\ref{pic:diff}б показан результат сравнения двух энергетических спектров (в качестве сравнительной метрики выбран логарифм отношения интенсивностей) для сфер из алюминия и урана диаметром 1~см. Как видим, даже в таких малых масштабах и малых (по сравнению с реальным пучком) интенсивностях можно регистрировать различия в энергетических спектрах.
    \end{itemize}
\begin{figure}[t]
    \begin{center}
        \begin{minipage}[h]{0.49\linewidth}
            \center{\includegraphics[width=60mm]{Sword.pdf} \\ а)}
        \end{minipage}
        \hfill
        \begin{minipage}[h]{0.49\linewidth}
            \center{ \includegraphics[width=60mm]{UranCube1.pdf} \\ б)}
        \end{minipage}         
        \vspace{-3mm}
        \caption{а) Опасный стальной предмет с неравномерной толщиной. б) Кубик урана в свинцовой оболочке (XY-распределение).}
    \end{center}
    \labelf{pic:sword}
    \vspace{-5mm}
\end{figure}
\begin{figure}[t]
    \begin{center}
        \begin{minipage}[h]{0.49\linewidth}
            \center{\includegraphics[width=60mm]{UranCube2.pdf}  \\ а)}
        \end{minipage}
        \hfill
        \begin{minipage}[h]{0.49\linewidth}
            \center{\includegraphics[width=60mm]{Hex.pdf} \\ б)}
        \end{minipage}
        \vspace{-3mm}
        \caption{а) Кубик урана в свинцовой оболочке (X-распределение). б) Сравнение целлюлозы и гексогена.}
    \end{center}
    \labelf{pic:hex}
    \vspace{-5mm}
\end{figure}
\begin{figure}[t]
    \begin{center}
        \begin{minipage}[h]{0.49\linewidth}
            \center{\includegraphics[width=60mm]{diffmat0.pdf} \\ а)}
        \end{minipage}
        \hfill
        \begin{minipage}[h]{0.49\linewidth}
            \center{\includegraphics[width=60mm]{diffmat.pdf}  \\ б)}
        \end{minipage} 
        \caption{а) Энергетические спектры различных материалов (общий вид).
        б) Энергетические спектры различных материалов (участок с энергией более 4~МэВ).}
    \end{center}
    \labelf{pic:diff0}
    \vspace{-5mm}
\end{figure}
\begin{figure}[t]
    \begin{center}
        \begin{minipage}[h]{0.49\linewidth}
            \center{\includegraphics[width=60mm]{diffmat1.pdf}  \\ а)}
        \end{minipage}
        \hfill
        \begin{minipage}[h]{0.49\linewidth}
            \center{\includegraphics[width=60mm]{Difference.pdf}   \\ б)}
        \end{minipage}
        \vspace{-3mm}
        \caption{а) Зависимость метрики от эффективного зарядового числа материала.
 б) Сравнение энергетических спектров из урановых и алюминиевых сфер.}
    \end{center}
    \labelf{pic:diff}
    \vspace{-5mm}
\end{figure}

Чтобы оценить возможность определения эффективного заряд материала по энергетическому спектру, было смоделировано сканирование шести мишеней из различных материалов (железо, свинец, алюминий, целлюлоза, олово, уран) с одинаковыми поперечными и различными продольными размерами. Продольный размер был выбран таким образом, чтобы общее ослабление потока гамма-излучения было одинаковым для всех материалов, и их нельзя было различить только путем анализа количества гамма-квантов пришедших в детекторы.

Энергетические спектры этих мишеней показаны на рисунках. Как видно, спектры для всех мишеней различны в области до 3~МэВ (см. рис.~\ref{pic:diff0}a) и в области после 4~МэВ (см. рис.~\ref{pic:diff0}б). Следует отметить, что это различие является значительным даже при малых интенсивностях электронного пучка ($10^8$ электронов), что указывает на более выраженное различие в реальном электронном пучке от ускорителя ($10^{15}$ электронов). Таким образом, можно сформулировать простой критерий, отличающий различные материалы: доля числа частиц с энергией больше 3~МэВ. Такой критерий позволяет отличать материалы по $Z_{eff}$ (см. рис~\ref{pic:diff}а). Следует отметить, что ошибки на рисунке~\ref{pic:diff} являются только статистическими и при интенсивностях, соответствующих реальному электронному пучку, будут незначительны.

Сформулированный критерий достаточно хорош для практического использования, однако он не является оптимальным решением, поскольку при его использовании большая часть информации о спектре теряется. В следующем разделе мы использовали простой пример, чтобы показать потенциал для создания трехмерной гамма-томографии с использованием полной информации о спектре.

\section*{Восстановление толщин материалов}
Рассмотрим одномерный случай, когда гамма-лучи проходят стопку из нескольких материалов с фиксированной общей толщиной, и нам нужно восстановить толщину отдельных материалов (см. схему~\ref{schema2}). Мы используем простую модель, в которой ослабление потока гамма-излучения задается следующим уравнением
\begin{equation}
\label{eq:gamma}
\frac{N(E)}{N_0(E)} = \exp(-\sum_i \Sigma^{mean}_i(E)x_i),
\end{equation}
где $x_i$ --- толщина $i$-слоя, $\Sigma^{mean}_i$ --- среднее макроскопическое сечение для группы материалов с близкими зарядовыми числами, $N,~N_0$ --- количество гамма-квантов. В этом случае мы не учитываем многократное рассеяние и наличие аннигиляционной линии. Мы считаем, что общая толщина известна и для восстановления толщины отдельных слоев мы используем метод наименьших квадратов, т. е. минимизируем такую сумму:
\begin{equation}
\sum_E(\ln \frac{N(E)}{N_0(E)} + \sum_i \Sigma^{mean}_i(E)x_i))^2 \to min
\end{equation}

Приведем пример работы алгоритма. Мы будем считать, что энергетическое разрешение составляет величину 10\%. Рассмотрим стопку из трех слоев: алюминиевого, железного и свинцового. Рисунок~\ref{rec:ex}а показывает вклад каждого восстановленного материала в общее ослабление потока гамма-лучей. Таблица~\ref{tab:rec} содержит результаты восстановления для данного примера. Как мы видим из таблицы, результат восстановления довольно точный.Чтобы прояснить возможности алгоритма, мы провели несколько численных экспериментов. Мы также как и в примере использовали алюминий, железо и свинец, и взяли около двухсот наборов с разным соотношение толщин слоев, причем суммарная толщина лежала в диапазоне от 30 до 180~сантиметров.
\begin{wrapfigure}[12]{r}{0.6\linewidth} 
    \includegraphics[width=\linewidth]{yed_schema_2.pdf}
    \vspace{-3mm}
    \caption{Восстановление толщин материалов (схема моделирования).}
    \labelf{schema2}
    \vspace{-5mm}
\end{wrapfigure}
Рисунок~\ref{rec:ex}б показывает разброс ошибки восстановления для данных наборов. Как мы можем видеть толщина тяжелых элементов определяется лучше всего: с точность порядка 5\%, а толщина элементов из группы железа хуже всего: величина ошибки достигает 30\%. Однако, мы рассматривали весьма простую модель и возникает вопрос: какая от неё польза? В данной модели мы использовали только энергетическое разрешение и по нему смогли провести восстановление послойной структуры объекта. При добавлении пространственного распределение, мы можем провести дополнительно сегментирование вдоль ещё одной оси и с учетом временной компоненты восстановить трехмерную структуру груза контейнера (3D гамма-томография). Таким образом наша простая модель показывает, что у нас есть перспектива создания действительно мощной системы для анализа содержимого контейнеров.

\begin{figure}[t]
    \begin{center}
        \begin{minipage}[h]{0.49\linewidth}
            \center{\includegraphics[width=60mm]{reconstruction.pdf} \\ а)}
        \end{minipage}
        \hfill
        \begin{minipage}[h]{0.49\linewidth}
            \center{\includegraphics[width=60mm]{relError.pdf}   \\ б)}
        \end{minipage}
        \vspace{-3mm}
        \caption{а) Вклад отдельных слоев в полное ослабление потока. б) Распределение ошибок восстановления для различных численных экспериментов.}
    \end{center}
    \labelf{rec:ex}
    \vspace{-5mm}
\end{figure}

\begin{table}
    \caption{Пример результата  работы алгоритма восстановления}
    \label{tab:rec}
\begin{center}
        \begin{tabular}[c]{|c|c|c|}
        \hline 
        Материал & Истинная толщина, см & Восстановленная, см \\ 
        \hline 
        Al & 20 & 19.6 \\ 
        \hline 
        Fe & 40 & 41.6 \\ 
        \hline 
        Pb & 30 & 28.7 \\ 
        \hline 
    \end{tabular}
\end{center}
\end{table}

\newpage
\section*{Измерение энергетического разрешения детектора}
В дополнение к моделированию, было измерено энергетическое разрешение сцинтилляционного детектора гамма-излучения. В качестве сцинтиллятора использовался кристалл BGO размером 10x30x100~мм (глубина кристалла подбиралась так, чтобы обеспечить полное поглощение гамма-квантов до 10~МэВ), для регистрации излучения сцинтиллятора использовался фотодетектор ArrayC-60035-4P. В качестве источников излучения использовались $^{22}Na$ (имеет две линии 0.511~МэВ и 1.275~МэВ) и $^{137}Cs$ (имеет линию 0.662~МэВ). Фотодетектор ArrayC-60035-4P  представляет из себя матрицу из четырёх фотодиодов размером 6x6~мм, оснащенную индивидуальным предусилителем с коэффициентом усиления равным 150. В процессе работы измерялся суммарный сигнал с двух фотодиодов матрицы. Сигнал c матрицы подавался на усилитель (ORTEC~579) и затем поступал на входной канал АЦП (CAEN~DT5742) и на дискриминатор (CAEN mod. 224), логический сигнал которого служил триггером в системе. В отсутствие источника шкала АЦП была прокалибрована в абсолютных единицах --- числе фотоэлектронов. В таблице~\ref{tab:ex} представлен результат измерения разрешения детектора ---  отношения СКО фотопика к его положению ($\frac{\sigma_E}{E}$). Световыход составляет 140 фотоэлектронов на МэВ, а порог шумов --- величину порядка 100~кэВ.\\
\begin{table}
    \caption{Измерение энергетического разрешения детектора}
    \label{tab:ex}
    \begin{center} 
        \begin{tabular}[c]{|c|c|c|}
            \hline 
            Источник & Энергия, МэВ & $\frac{\sigma_E}{E}$\\
            \hline 
            $^{22}Na$&0.511 & 19.0 \%  \\ 
            \hline 
            $^{137}Cs$&0.662 & 14.7\%\\ 
            \hline 
            $^{22}Na$& 1.275 & 13\% \\
            \hline 
        \end{tabular} 
    \end{center}
\end{table}


\section*{Выводы}
Результаты:   
    \begin{enumerate}
        \item Энергетические спектры материалов с различным эффективным зарядовым числом различаются в областях энергий до 3 и после 4~МэВ. Оценена зависимость зарядового числа от доли гамма-квантов с энергией выше 3~МэВ. Данный метод позволяет идентифицировать в контейнере отдельные группы элементов по зарядовому числу (легкие, средние, тяжелые) в одной экспозиции при одной фиксированной энергии электронов, оптимально вблизи 8~МэВ. При этом требуется разрешение не хуже 15\% и эффективность регистрации около 90\%. 
        \item Энергетическое разрешение порядка 10\% позволяет определить толщину отдельного слоя в многослойной структуре с точностью 25\%.
        \item Измерено энергетическое разрешение детектора на основе BGO, в целом полученные результаты говорят о высоких эксплуатационных характеристиках и качестве материала. Представляется возможным достижение характеристик заявленных производителем ($FWHM \sim 9\%$ для $^{137}Cs$).
    \end{enumerate}

Развитие данной тематике является перспективным направлением деятельности, но требует финансовой поддержки, при наличии которой становиться возможным разработка программы для проверки содержимого транспортного контейнера по заявленному манифесту, и создание программы для гамма-томографии содержимого контейнеров.

Данная работа частично профинансирована в рамках госзадания № 3.3008.2017/ПЧ Министерства Образования и Науки Российской Федерации, и частично выполнена при поддержке гранта РНФ № 16-12-10039.
\bibliographystyle{pepan}
\bibliography{references.bib}

\end{document}
