%% Согласно ГОСТ Р 7.0.11-2011:
%% 5.3.3 В заключении диссертации излагают итоги выполненного исследования, рекомендации, перспективы дальнейшей разработки темы.
%% 9.2.3 В заключении автореферата диссертации излагают итоги данного исследования, рекомендации и перспективы дальнейшей разработки темы.
\begin{enumerate}
  \item Дано описание принципов моделирование с помощью транспортного кода GEANT4, а также обсуждены особенности моделирования электромагнитных взаимодействий в контексте  данной работы.
  \item Написана реализация метода статистической регуляризации Турчина для решения обратных задач, причем этом метод был обобщен за счет рассмотрения восстанавливаемый сигнала как вектора в функциональном базисе.
  \item Спроектирован компактный орбитальный детектор протонов и электронов, проведены расчеты массо-габаритных характеристик детектора. Для детектора были разработаны методики восстановления сигнала в одночастичном и интегральном режимах работы. По итогам моделирования ожидается разрешающая способность по энергии не менее 2\% для 50 МэВ протонов. Показано возможность применять  метод статистической регуляризации Турчина для восстановления спектра частиц в интегральном режиме работы.
  \item Разработана методика анализа содержимого транспортного контейнера на основе энергетических спектром материалов. Показано что такая методика может служить альтернативой существующему методу дуальных энергий. В частности установлено что энергетические спектры материалов с различным эффективным зарядовым числом различаются в областях энергий до 3 и после 4~МэВ. Оценена зависимость зарядового числа от доли гамма-квантов с энергией выше 3~МэВ. Данный метод позволяет идентифицировать в контейнере отдельные группы элементов по зарядовому числу (легкие, средние, тяжелые) в одной экспозиции при одной фиксированной энергии электронов, оптимально вблизи 8~МэВ. При этом требуется разрешение не хуже 15\% и эффективность регистрации около 90\%. Также показано что энергетическое разрешение порядка 10 \% позволяет определить толщину отдельного слоя в многослойной структуре с точностью 25\%.
  \item Проведено моделирования числа частиц в лавине убегающих электронов, показано важность учета рождения гамма-квантов и позитронов при расчетах числа частиц. Были показано что расчеты выполнение в работе (Орешкин) дают слишком большую оценку числа частиц.
  \item Проведено исследование механизмов гамма и позитронной обратной связи предложенных в работе (Дуайер). Показано что хотя и при выполнении некоторых условий механизм действительно может эффективно работать, эти условия более ограничены. Были разобраны недостатки модели Дуайера и проведено моделирование для уточнения влияния механизмов, которое показало что неучтенные Дуайером факторы уменьшают эффективность модели и в реальных условиях эти механизмы не способны дать взрывной рост числа частиц.
  %\item В работе обсуждено характер роста числа частиц в лавине убегающих электронов, показано большая важность размеров облака перед величиной поля. Обсуждено влияние этого наблюдения на работу механизмов обратной связи и стримерных механизмов.
  \item По результатам моделирования доказана трипольная структура грозовых облаков отвечающих за явления LLLE TGE и HEP TGE наблюдаемые на научной станции на г. Арагац.
  \item Представлен прототип "реакторной" модели процессов в грозовом облаке, показано что модель хорошо адаптируется к различным природным условиям и в зависимости от них может описывать как явления TGE, так и TGF, а также предсказывает ряд других экспериментально наблюдаемых эффектов. 
  \item Проведено моделирование рождения нейтронов в грозовых облаках в рамках проекта ЧИБИС, показана не целесообразность разработки орбитального детектора нейтронов
\end{enumerate}
