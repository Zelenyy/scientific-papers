\documentclass[]{article}

%opening
\title{}
\author{}

\begin{document}

\maketitle

\begin{abstract}

\end{abstract}


\paragraph{Slide 1} Hello, I'm Mikhal Zelenyy,  today I am going to talk about services for condition database on BM@N  experiment.

\paragraph{Slide 2} On slide general scheme of NICA complex and one of  its part is BM@N.  Barion matter at Nuclotron is experiment for  research production of strange matter in fixed-target heavy-ion collisions. 

\paragraph{Slide 3} For experiment needs complex information system which presented on slide.  It include system for data acquisition, databases and data processing of stored data. In this report I want tell about Condition Database. In this database contain general information about each run of experiment according.

\paragraph{Slide 4} Condition database used for getting general information by run, it don't contain extend information about run events (This role invoke metada database) and necessary for fast search run with heeds parameters. Typically parameters is particle type and energy, experiment time and run statistics.

\paragraph{Slide 5} For condition database developed set of services: 


\paragraph{Slide 6} Nextly, I  speak in more detail about one more service. BM@N experiment worked  with 2015 year and have old data which needed in migration to actual database. For solution of this task, was developed small application. Smart Data Parser  take user data in different formats and  loaded  to database using data model described in JSON. In current time application work with CSV and XML files. can be work with different data base and has CLI and Qt-based GUI. 

\paragraph{Slide 7} For communication with database is used SqlAlchemy framework. It work with different drivers, using unification Python Database API 2.0. By default our application used psycorg driver for connection to postgresql, but can be work with another database just need install necessary driver. SQlAlcheme allow get metadata from database for validate input JSON description and loading parsed data to database.

\paragraph{Slide 8} For describe of input data, used description in JSON format which include information about target database, parser preferences and structure of input data. For validation user description used JSON Schema, which also  allow generated documentation and insert default value in description.

\paragraph{Slide 9} For example, consider generation templates of scheme and loading  to database. Database have next tables, our script for every table generate json file which contain next description.

\paragraph{Slide 10} Using this description we can load CSV data to database and check result using pgAdmin

\paragraph{Slide 11} Also except CLI scripting was developed GUI? which allow ...

\paragraph{Slide 12} For GUI creation we use PySide ... . 
 What an interesting moment, for creation of description editor used JSON Schema and database metadata. Input fields generated by schema, also some field has connection to database and choosing in it limited by database metadata.


\end{document}
